
%% bare_conf.tex
%% V1.4
%% 2012/12/27
%% by Michael Shell
%% See:
%% http://www.michaelshell.org/
%% for current contact information.
%%
%% This is a skeleton file demonstrating the use of IEEEtran.cls
%% (requires IEEEtran.cls version 1.8 or later) with an IEEE conference paper.
%%
\documentclass{IEEEtran}
\usepackage[spanish]{babel}
\usepackage[utf8]{inputenc}
\usepackage{enumerate}
\usepackage{cite}
\usepackage{graphicx}  
\usepackage{subfig}

% correct bad hyphenation here
\hyphenation{op-tical net-works semi-conduc-tor}


\begin{document}
%
% paper title
% can use linebreaks \\ within to get better formatting as desired
% Do not put math or special symbols in the title.
\title{Práctica 1 - Difuminado de imagen POSIX}


% author names and affiliations
% use a multiple column layout for up to three different
% affiliations
\author{\IEEEauthorblockN{Ronald Alexander Sarmiento Galviz\\}
\IEEEauthorblockA{Universidad Nacional Bogotá, Colombia \\
Email:roasarmientoga@unal.edu.co\\}
}


% make the title area
\maketitle

% As a general rule, do not put math, special symbols or citations
% in the abstract
\begin{abstract}
Se implementa un algoritmo de difuminado de imagen para paralelizar 
y evaluar en varios casos de prueba su eficiencia. Se concluye que debido 
a las caracteristicas del equipo es mas eficiente paralelizar, usualmente
no suman eficiencia lanzar mas de 4 hilos.
\end{abstract}

% no keywords




% For peer review papers, you can put extra information on the cover
% page as needed:
% \ifCLASSOPTIONpeerreview
% \begin{center} \bfseries EDICS Category: 3-BBND \end{center}
% \fi
%
% For peerreview papers, this IEEEtran command inserts a page break and
% creates the second title. It will be ignored for other modes.
\IEEEpeerreviewmaketitle



\section{Introducción.}


\subsection{Difuminado de imagen (Blur)}
El algoritmo de difuminado de imagen consiste en intercambiar 
cada pixel por un promedio de los valores RGB que se encuentran en 
los pixeles adyacentes a él. El nivel de difuminado, o kernel determina
 la aplitud del rango que se toma para evaluar el promedio. Es decir, 
 si se escoge un kernel de 5 significa que cada pixel será cambiado por 
 el promedio que hay en los pixeles de la matriz 5 x 5 cuyo centro es el 
 pixel en cuestión.


\section{Paralelización del algoritmo.}
Para distribuir la carga se tomó en ancho de la imagen de entrada
 y se dividió en el numero de hilos, asi cada hilo procesa una 
 imagen de $(width/THREADS) x \ height$.



\section{Experimentos y resultados.}
EL algoritmo se probó usando variando el kernel y numero de hilos para
3 tipos de imagenes (720px , 1080 px y 4K). Con los valores 
$kernel:{4,6,8,10,12,14} \ hilos:{2,4,8,16} $
\\

\begin{figure}[h]
\centering
\includegraphics[width=4.0in]{imagen-720px.png}
\caption{Speedup vs kernel - imagen 720px }
\label{fig_sim}
\end{figure}

\begin{figure}[!t]
\centering
\includegraphics[width=4.0in]{imagen-1080px.png}
\caption{Speedup vs kernel - imagen 1080px }
\label{fig_sim}
\end{figure}

\begin{figure}[!t]
\centering
\includegraphics[width=4.0in]{imagen-4K.png}
\caption{Speedup vs kernel - imagen 4K }
\label{fig_sim}
\end{figure}

Las siguientes tablas muestran el Speedup para los casos de cada imagen.
\begin{table}[h]
\begin{tabular}{lllll}
\hline
720             & Speedup          &                  &                  &                   \\ \hline
\textbf{Kernel} & \textbf{2 Hilos} & \textbf{4 Hilos} & \textbf{8 Hilos} & \textbf{16 Hilos} \\
4               & 0,8174157303     & 1,141176471      & 1,106463878      & 1,093984962       \\
6               & 1,013333333      & 1,288135593      & 1,301369863      & 1,310344828       \\
8               & 1,257978723      & 1,367052023      & 1,351428571      & 1,351428571       \\
10              & 1,365296804      & 1,472906404      & 1,483870968      & 1,469287469       \\
12              & 1,49112426       & 1,558762887      & 1,558762887      & 1,5782881         \\
14              & 15,32786885      & 1,623263889      & 1,620450607      & 1,612068966       \\ \hline
\end{tabular}
\caption{Valores Speedup para imagen 720px}
\end{table}

\begin{table}[h]
\begin{tabular}{lllll}
\hline
1080            & Speedup          &                  &                  &                   \\ \hline
\textbf{Kernel} & \textbf{2 Hilos} & \textbf{4 Hilos} & \textbf{8 Hilos} & \textbf{16 Hilos} \\
4               & 1,311764706      & 1,319526627      & 1,363914373      & 1,347432024       \\
6               & 1,430523918      & 1,457076566      & 1,463869464      & 1,460465116       \\
8               & 1,803691275      & 1,909413854      & 1,8728223        & 1,933453237       \\
10              & 1,940414508      & 2,083449235      & 1,973649539      & 2,01615074        \\
12              & 1,869035533      & 1,998914224      & 1,994582882      & 1,992424242       \\
14              & 1,76607717       & 1,892334195      & 1,895599655      & 1,905464007       \\ \hline
\end{tabular}
\caption{Valores Speedup para imagen 1080px}
\end{table}

\begin{table}[h]
\begin{tabular}{lllll}
\hline
4K              & Speedup          &                  &                  &                   \\ \hline
\textbf{Kernel} & \textbf{2 Hilos} & \textbf{4 Hilos} & \textbf{8 Hilos} & \textbf{16 Hilos} \\
4               & 1,617977528      & 1,746967071      & 1,730967373      & 1,751013318       \\
6               & 1,894245723      & 2,067911715      & 2,100724388      & 2,089910776       \\
8               & 1,648994841      & 2,048850575      & 2,039832746      & 2,04839779        \\
10              & 1,932943795      & 2,0839958        & 2,001728858      & 1,965483095       \\
12              & 1,795095635      & 1,939692634      & 1,900415369      & 1,920558296       \\
14              & 1,842920354      & 1,891699295      & 1,877438408      & 1,818777293       \\ \hline
\end{tabular}
\caption{Valores Speedup para imagen 4K}
\end{table}


\section*{Conclusiones.}
La practica muestra que el algoritmo de difuminado es paralelizable, sin embargo
demuestra que la eficiencia de la paralelización tiene un pico facilmente acotable
por lo que para calculos extensos no es necesario paralelizar con hilos a un numero mayor al
de los cores del equipo.




% that's all folks
\end{document}


